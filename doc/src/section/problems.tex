\section{Problems}

%TODO add more problems

During the course of the project, we encountered several challenges that required us to find practical solutions. The main issues were:

\begin{itemize}
    \item \textbf{Hardware selection and procurement:}
    Since we decided to use a commercial vehicle kit instead of the equipment provided by the institute, we had to handle the procurement ourselves. It took some time to choose a suitable kit, as we first had to verify compatibility with the Raspberry Pi we planned to use. Additionally, finding and ordering the right batteries for power supply required further research. Delays in the delivery of components pushed back our project timeline.

    \item \textbf{Limited testing capabilities:}
    As we only had one Raspberry Pi and one vehicle kit, which were mounted together, testing was limited to one team member at a time. This made it more difficult to verify newly developed software features, as only the person currently in possession of the hardware could run tests. The rest of the team had to rely on that member to verify their work.

    \item \textbf{Power connection issues:}
    The battery holder with the included DC plug, used to power the Raspberry Pi shield, had an intermittent contact issue that was difficult to locate. This caused unexpected power losses and system shutdowns during development and testing, making debugging and progress more difficult.

    \item \textbf{Latency:}
    Using MQTT for the camera feed introduced additional latency and load, resulting in extremely low FPS and delayed images.
    To fix this, we decided to use websockets instead, as they caused no additional latency.
\end{itemize}